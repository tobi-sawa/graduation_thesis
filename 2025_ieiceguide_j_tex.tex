\documentclass[twocolumn,a4paper,dvipdfmx]{ieicejsp}
\usepackage[T1]{fontenc}
\usepackage{lmodern}
\usepackage{textcomp}
\usepackage{latexsym}
%\usepackage[fleqn]{amsmath}
%\usepackage{amssymb}
\usepackage{float}
\usepackage{graphicx}
\usepackage[section]{placeins}

\title{{\bf LLMマルチエージェントを用いたWebサービス合成}
  {\normalsize \\LLM-based Multi-Agent Approach to Web Service Composition} }
  \author{
    飛澤佑季$^1$ \\ Yuuki Tobisawa \and
    村上陽平$^1$ \\ Yohei Murakami
  }
  \affliate{%
    立命館大学$^1$\\
    Ritsumeikan Univ.\\
}

\begin{document}
\maketitle

\section{まえがき}
近年,Web サービスは生活や産業の基盤となり,複数サービスを組み合わせるサービス合成の重要性が増している.しかし従来手法では,ユーザ要求を時相論理・述語論理などで記述する必要があり,非専門家にとって大きな障壁となっている.そこで本研究は,自然言語で要求を扱える大規模言語モデル(LLM) に着目する.ただし,LLMの入力コンテキスト長の制限により,単一エージェントでは多数のサービス記述を同時に考慮した複合サービスの推薦が困難であるため,契約ネットプロトコル(Contract Net Protocol;CNP)に基づくマルチエージェント協調により,プロンプト制約を緩和しつつ高精度なサービス合成を実現する.

\section{CNPに基づくAPI選定フレームワーク}
本研究の提案手法は,CNPに基づき,自然言語で記述された要求からサービス推薦・合成を可能にする枠組みである.図 1のように統括役のManagerが要求文を受け取り,必要となる処理をタスク候補として整理し,タスク公告を行う.次にContractorが公告を参照し,自身が担当可能なタスクについて,担当可能である理由を入札として返す.ここで各ContracterはAPIに特化した知識に基づいて判断するため,単一エージェントで生じやすい仕様理解不足や知識希薄化,プロンプト肥大を抑制できる.マネージャは入札結果を統合してタスクごとのAPIを決定する.最後に,確定したタスク集合・採用されたAPI・依存関係に基づき,API群を出力する.

\begin{figure}[!b]
  \centering
  \includegraphics[width=\linewidth]{cnp_image.png}
  \caption{CNPに基づくAPI選定フレームワーク}
  \label{fig:cnp_image}
\end{figure}

\FloatBarrier
\section{推論タスクの責任分担}
本研究では,複数ステップに分解して逐次的に実行するとともに,各ステップに目的・参照情報・出力形式を与えて推論を誘導する Guided Reasoning Chains の考え方に基づき,サービス合成推論をLLMの内部推論としてではなく,外部化可能な推論タスクの系列として捉える.その上で,CNP上での責任分担を設計・比較する.
具体的に,サービス合成推論を3ステップ(コア機能分解・カテゴリマッピング・API選択)として定義し\cite{1},各ステップを推論タスクとして切り出す.
これらの推論タスクの分担パターンとしてコア機能分割,カテゴリマッピングをManaeger,API選択をContractorで行うマネージャ主導型,全てのステップをContractorで行うコントラクタ主導型,コア機能分割をManeger,カテゴリマッピングとAPI選択をContracterで行う協調型の3方式を提案する.そしてProgrammableWeb由来のデータを用い,API推薦結果に対して再現率・適合率・F1値を算出して比較する.併せて,推論コストとしてトークン数および実行時間も計測し,3方式間で総合的に比較評価する.


\section{おわりに}
本研究の貢献は,CNPに基づくマルチLLMエージェント協調により,要求の粒度や曖昧さに応じてタスク構造と依存関係を入札・応札過程で調整しつつ安定したサービス合成計画を生成する枠組みを提示する点にある.また,その3ステップをCNP上でどのように分担すべきかを検証するために,マネージャ主導・コントラクタ主導・協調型の3類型に分類して比較可能にした.それによってAPIカテゴリ特化エージェントによる役割分担によってプロンプト肥大と知識希薄化を抑えられ,大規模サービス集合に対しても高精度かつスケーラブルなサービス推薦・合成を可能にする点にあると考える.

\section*{謝辞}
本研究はJSPS科研費JP25K03227の助成を受けたものである.

\begin{thebibliography}{9}
\bibitem{1}
Kenji Matsumoto and Yohei Murakami, 'Guided Reasoning Chains for API Recommendation',SOC4AI,2025-12-2.
\end{thebibliography}
\end{document}
