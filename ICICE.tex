% !TeX program = ptex2pdf
% !TeX encoding = UTF-8
\documentclass[a4paper,10pt,twocolumn]{jsarticle}
% --- packages ---
\usepackage{newtxtext,newtxmath} % 文字を少し論文っぽく(不要なら削除OK)
\usepackage{graphicx}
\usepackage{url}
\usepackage{enumitem}
\setlist{nosep}

% 余白(必要に応じて調整)
\usepackage[top=18mm,bottom=18mm,left=16mm,right=16mm]{geometry}
\setlength{\columnsep}{7mm}

\title{LLMマルチエージェントを用いたWebサービス合成\\
\large LLM-based Multi-Agent Approach to Web Service Composition}

\author{%
\begin{tabular}{cc}
飛澤 佑季 & 村上 陽平\\
Yuuki Tobisawa & Yohei Murakami\\
立命館大学 & 立命館大学\\
Ritsumeikan Univ. & Ritsumeikan Univ.\\
\end{tabular}
}

\date{} % 日付を出さない

\begin{document}
\maketitle

\section{まえがき}
近年,Webサービスは生活や産業の基盤となり,複数サービスを組み合わせるサービス合成の重要性が増している.しかし従来手法では,ユーザ要求を時相論理・述語論理などで記述する必要があり,非専門家にとって障壁が大きい.
本研究は,要求を自然言語で扱えるLLMに着目し,CNPに基づくマルチエージェント協調により,プロンプト制約を緩和しつつ高精度なサービス合成を実現する.

\section{提案手法}
本研究の提案手法は,契約ネットプロトコル(Contract Net Protocol; CNP)に基づき,自然言語で記述されたユーザ要求からサービス合成の実行可能な計画(Plan)を生成する枠組みである.まず統括役のマネージャ(Manager)エージェントが要求文を受け取り,意図と達成条件を解釈して,必要となる処理をタスク候補として整理し,タスク公告を行う.次に,複数のコントラクタ(Contractor)エージェントが公告を参照し,自身が担当可能なタスクについて,必要条件や想定手順,適用可能なAPI,担当可能である理由を入札として返す.ここで各コントラクタはAPIカテゴリ/機能領域に特化した知識に基づいて判断するため,単一エージェント方式で生じやすい仕様理解不足や知識希薄化,およびプロンプト肥大を抑制できる.マネージャは入札結果を統合してタスクごとの担当(採用API)を決定し,必要に応じてタスク境界や依存関係を調整してタスク構造を安定化させる.最後に,確定したタスク集合・採用API・依存関係に基づき,APIまたはAPI群を出力する.

\section{研究の課題と解決}
本研究における主要課題は,(1) 自然言語要求に対する適切なタスク分解,および (2) LLMエージェントの役割分担の設計,の二点である.
課題(1)に対しては,要求の粒度や曖昧さが一定でないことに起因して,単一の分解方針では過分解・不足分解が発生し,タスク境界や依存関係が不安定になりやすいという問題がある.そこで本研究では,契約ネットプロトコル(CNP)の考え方を応用し,タスク分解における「分解の主導権」と「調整の仕組み」を状況に応じて切り替えられる枠組みを導入する.具体的には,(i) マネージャ主導,(ii) コントラクタ主導,(iii) 協調型の三つの分解プロトコルを用意し,要求の性質に応じて段階的に適用することで,いずれの方式でも一貫したタスク構造を得るとともに,入札・応札過程でタスク境界や前後関係を調整することで計画破綻を抑制し,安定したサービス合成を実現する.
課題(2)に対しては,対象とするサービス領域が広がるほど単一エージェントでは仕様理解が浅くなり,さらに専門知識を網羅的に与えるとプロンプトが肥大化するという問題がある.これに対し,本研究ではLLMエージェントをAPIカテゴリごとに専門化し,各エージェントが担当領域に限定された知識のみを参照して推論・入札を行うように設計する.この役割分担により,プロンプト制約を緩和しつつ,仕様理解に基づく高精度な推薦と,大規模サービス集合に対するスケーラビリティを両立する.また,タスク分解プロトコルの違いが情報共有や推論過程に与える影響を比較することで,役割分担設計の妥当性を分析可能とする.

\section{研究の貢献}
本研究は,Shared Plans の主要概念をクロスドメイン QA 向けに定式化し,複数 LLM エージェントが共通計画を共有・更新しながら分担・合意・実行するマルチエージェントシステムとして実装した.また HotpotQA で Plan-and-Solve 等と比較し,EM で XX ポイント,F1 で YY ポイントの向上を確認した.Shared Plans を,計画構築・タスク追加・合意・実行の各段階で更新される実行時表現として具現化し,理論を動作する協調プロトコルへ翻訳した点に新規性があると考える.

\begin{thebibliography}{9}
\bibitem{grosz1996}
B. J. Grosz, S. Kraus,
``Collaborative Plans for Complex Group Action,''
\textit{Artificial Intelligence},
vol. 86, no. 2, pp. 269--357, 1996.
\end{thebibliography}

\end{document}
