\PassOptionsToPackage{dvipdfmx}{graphicx}
\PassOptionsToPackage{dvipdfmx}{xcolor}
\documentclass{kuisthesis} % 特別研究報告書
%\documentstyle[master]{kuisthesis}		% 修士論文(和文)
%\documentstyle[master,english]{kuisthesis}	% 修士論文(英文)

\usepackage[dvipdfmx]{graphicx}
\usepackage{float}
\usepackage{listings}
\usepackage[dvipdfmx]{xcolor}
\usepackage{amsmath}

\lstdefinelanguage{json}{
  basicstyle=\ttfamily\small,
  numbers=left,
  numberstyle=\tiny,
  stepnumber=1,
  numbersep=8pt,
  showstringspaces=false,
  breaklines=true,
  frame=single,
  tabsize=2
}

\def\LATEX{{\rm (L\kern-.36em\raise.3ex\hbox{\sc a})\TeX}}
\def\LATex{\iLATEX\small}
\def\iLATEX#1{L\kern-.36em\raise.3ex\hbox{#1\bf A}\kern-.15em
    T\kern-.1667em\lower.7ex\hbox{E}\kern-.125emX}
\def\LATEXe{\ifx\LaTeXe\undefined \LaTeX 2e\else\LaTeXe\fi}
\def\LATExe{\ifx\LaTeXe\undefined \iLATEX\scriptsize 2e\else\LaTeXe\fi}
\let\EM\bf
\def\|{\verb|}
\def\<{\(\langle\)}
\def\>{\(\rangle\)}
\def\CS#1{{\tt\string#1}}

\jtitle[契約ネットプロトコルに基づく\\マルチエージェントシステムを用いた複合サービス推薦]%	% 和文題目(内容梗概/目次用)
	{契約ネットプロトコルに基づく\\マルチエージェントシステムを用いた\\複合サービス推薦}	% 和文題目
\etitle{Composite Service Recommendation Using Contract-Net-Based LLM Multi-Agent Systems}	% 英文題目
\jauthor{飛澤 佑季}				% 和文著者名
\eauthor{Yuki Tobisawa}			% 英文著者名
\supervisor{村上 陽平 教授}		% 指導教官名
\date{2025年1月9日}				% 提出年月日
\department{社会情報学}				% 修士論文の場合の専攻名

\begin{document}
\maketitle					% 「とびら」の出力

\begin{jabstract}				% 和文梗概
近年,Webサービスは生活や産業のあらゆる場面で不可欠な存在となり,複数のWebサービスを組み合わせて利用するサービス合成の重要性が高まっている.しかし従来手法では,ユーザーの目的や要求を時相論理・述語論理といった形式的表現で記述する必要があり,専門知識を持たない利用者にとって大きな障壁となっていた.

このような問題を解決するために,大規模言語モデル(LLM)を用いて,自然言語で記述された要求からそのゴールを満たすWebサービスを推論するエージェントが提案されてい.しかしながら,単一エージェント方式では,多数のAPI仕様や専門知識を1つのプロンプトに集約する必要があり,プロンプト肥大化や入力制限により推論精度が低下しやすい.

そこで,複数のLLMエージェントを協調させるマルチエージェントシステムを導入し,エージェントにAPI仕様を分散保持させることでプロンプト制約を緩和する.具体的には,契約ネットプロトコル(Contract Net Protocol: CNP)に基づき,マネージャが要求を公告し,コントラクタエージェントが担当可否とAPI案を入札,マネージャが統合して合成計画を確定する.

本手法の実現にあたり,取り組むべき課題は以下の2点である.
\begin{description}
\item[推論タスク割り当て]
自然言語で記述されたユーザ要求に対し,CNPに基づいてタスク分解を行う際に,要求の粒度や曖昧さに応じて分解方式をどのように選択すべきか,またその分解をマネージャ等のどの役割のエージェントが担うことが適切かを検証する必要がある.

\item[コントラクタエージェントの組織化]
CNPでは,コントラクタエージェント数の増加に伴い公告・入札の通信量が増大し,スケーラビリティが低下しやすいという課題がある.このデメリットを緩和するため,コントラクタエージェントに対してどのAPIを割り当てるべきか,またコントラクタエージェント群のカテゴリ別階層化を検討し,組織構造の違いが入札の精度や合成結果に与える影響を検証する必要がある.
\end{description}

1つ目の課題に対しては,マネージャ主導・コントラクタ主導・協調型の3つのタスク分解プロトコルを比較する.マネージャ主導では,マネージャがタスク分解とカテゴリ分解を行ってタスク仕様を公告し,コントラクタエージェントは担当可否とAPI案を入札する.
コントラクタ主導では,コントラクタエージェントが自身の担当API仕様に基づいてタスク分解とカテゴリマッピングを行い入札する.
協調型では,マネージャがタスク分解を提示したうえで,コントラクタエージェントが分解されたタスクに対してカテゴリマッピングを行って入札する.
本研究では,これら3つのプロトコルの分担の違いが合成計画の整合性および計画生成精度に与える影響を分析する.

2つ目の課題に対しては,コントラクタエージェントの組織化により解決を図る.具体的には,コントラクタエージェントに割り当てる担当として,単一APIを担当させる設計に加え,APIカテゴリ単位で複数APIをまとめて担当させる設計も比較対象とする.各エージェントは,割り当てられたAPIの仕様のみを参照して推論・入札を行うようにし,推薦精度に与える影響を検証する.

本研究では,提案手法の有効性を検証するため,ProgrammableWeb の複合サービスデータを用いた評価を行った.具体的には,複合サービスデータ 400 件をテストデータとし,対応する 909 件の Web API を対象として,マネージャ主導・コントラクタ主導・協調型という3種類のタスク分解方式を適用してサービス合成を実行し,各方式の計画生成精度を比較した.生成されたサービス計画と正解計画を照合し,適合率,再現率,F1 スコアを算出することでタスク分解方式の性能を評価した.また,コントラクタエージェントの組織化(単一API単位の割当と,カテゴリ単位で複数APIを割り当てる設計)についても,各エージェントの入札行動と推薦精度を比較し,役割分担の違いがサービス合成結果に与える影響を分析した.また本研究の貢献は以下の通りである.
\begin{description}

\item[推論タスク割り当て]
マネージャ主導・コントラクタ主導・協調型の3つのタスク分解プロトコルを比較し,ユーザ要求の粒度に応じた適切な方式を分析した。適合率・再現率・F1 値で評価した結果,本手法は単一方式を上回る性能を示した。

\item[コントラクタエージェントの組織化] 
エージェント数と担当API数を変化させた実験を実施し,専門性の深さと分担効率の関係を評価した。その結果,APIカテゴリを限定した専門エージェントを用いることで,単一エージェント方式と比較してサービス推論の正確性が向上し,大規模サービス集合に対してもスケーラブルに動作することが明らかとなった
\end{description}
\end{jabstract}


\begin{abstract}
In recent years, Web services have become indispensable in all aspects of daily life and industry, and the importance of service composition—combining multiple Web services—has been increasing. However, conventional methods require users to describe their objectives and requirements using formal expressions such as temporal logic or predicate logic, which presents a significant barrier for users without specialized knowledge.

To address this problem, agents that use Large Language Models (LLMs) to infer Web services satisfying goals from requirements described in natural language have been proposed. However, in single-agent approaches, numerous API specifications and domain knowledge must be consolidated into a single prompt, which tends to reduce inference accuracy due to prompt bloating and input limitations.

Therefore, we introduce a multi-agent system that coordinates multiple LLM agents, allowing agents to distributedly maintain API specifications to alleviate prompt constraints. Specifically, based on the Contract Net Protocol (CNP), a manager announces requirements, contractor agents bid on their capability to handle tasks along with API proposals, and the manager integrates these to finalize the composition plan.

In realizing this approach, there are two challenges to be addressed:

\begin{description}
\item[Inference Task Allocation]
When decomposing tasks based on CNP for user requirements described in natural language, it is necessary to verify how the decomposition method should be selected according to the granularity and ambiguity of requirements, and which role of agent (such as the manager) should appropriately handle this decomposition.

\item[Organization of Contractor Agents]
In CNP, as the number of contractor agents increases, the communication volume for announcements and bids grows, which tends to reduce scalability. To mitigate this disadvantage, it is necessary to examine which APIs should be assigned to contractor agents, consider category-based hierarchization and aggregation of contractor agent groups, and verify how differences in organizational structure affect bidding accuracy, composition results, and communication load.
\end{description}

For the first challenge, we compare three task decomposition protocols: manager-led, contractor-led, and collaborative. In the manager-led approach, the manager performs task decomposition and category decomposition, announces task specifications, and contractor agents bid on their capability to handle tasks along with API proposals. In the contractor-led approach, contractor agents perform task decomposition and category mapping based on their assigned API specifications and submit bids. In the collaborative approach, the manager presents task decomposition, and contractor agents perform category mapping for the decomposed tasks and submit bids. This research clarifies which of these three protocols is suitable according to requirement types such as granularity and ambiguity, and analyzes how differences in task distribution affect the consistency of composition plans and plan generation accuracy.

For the second challenge, we seek solutions through organization of contractor agents. Specifically, in addition to a design where each contractor agent handles a single API, we also compare a design where agents handle multiple APIs grouped by API category. Each agent performs inference and bidding by referencing only the specifications of their assigned API (or multiple APIs belonging to their category), and we verify the effects on recommendation accuracy and communication load.

To validate the effectiveness of the proposed method, we conducted evaluations using composite service data from ProgrammableWeb. Specifically, using 400 composite service data items as test data and targeting 909 corresponding Web APIs, we executed service composition applying three types of task decomposition methods—manager-led, contractor-led, and collaborative—and compared the plan generation accuracy of each method. By comparing generated service plans against ground truth plans and calculating precision, recall, and F1 scores, we evaluated the performance of the task decomposition methods. Additionally, regarding the organization of contractor agents (single API unit assignment versus assigning multiple APIs per category), we compared the bidding behavior and recommendation accuracy of each agent, analyzing how differences in role distribution affect service composition results. The contributions of this research are as follows:

\begin{description}
\item[Inference Task Allocation]
We compared three task decomposition protocols—manager-led, contractor-led, and collaborative—and analyzed appropriate methods according to the granularity of user requirements. Evaluation results using precision, recall, and F1 scores demonstrated that our method outperforms single-method approaches.

\item[Organization of Contractor Agents]
We conducted experiments varying the number of agents and assigned APIs, evaluating the relationship between depth of specialization and distribution efficiency. The results revealed that using specialized agents limited to specific API categories improves the accuracy of service inference compared to single-agent approaches and operates scalably even for large-scale service collections.
\end{description}

\end{abstract}

\tableofcontents				% 目次の出力

\section{はじめに}\label{sec-intro}		% 本文の開始
近年,サービスコンピューティングの発展により,多種多様な Web サービスを組み合わせて新たな付加価値を提供する複合サービスが盛んに構築されている。
翻訳サービスと音声認識サービスを連携させたリアルタイム音声翻訳や,地理情報サービスと天気情報サービスを統合した地図表示サービスなど,個々の Web サービスでは実現できない機能を柔軟に実装できる。しかし,公開されている Web サービスのうち,複合サービスに実際に利用されているものは一部にとどまり,ユーザが膨大な候補の中から目的に適したサービスを選択することは依然として大きな負担となっている。

従来の複合サービス合成は,水平型サービス合成と垂直型サービス合成の二つのアプローチに大別される。前者は,所与のワークフローに対して各タスクを実行するのに適した Web サービスの組み合わせを選択する枠組みであり,ワークフロー設計のコストが高いという課題を持つ。後者は,人工知能のプランニング技術を用いて,論理式で与えられたゴール状態に到達するサービス実行系列を生成する手法であるが,ユーザが時相論理や述語論理といった形式的表現で要求を記述しなければならず,エンドユーザにとって扱いやすいとは言い難い。この問題を軽減するため,自然言語で記述された要件定義からサービスの組み合わせを推定する手法や,WSDL やサービス説明文,サービスネットワークをテキストマイニングやグラフ埋め込みによって特徴ベクトル化し,機能ごとにクラスタリングする手法が提案されてきた。

一方,近年急速に発展している大規模言語モデル(Large Language Model; LLM)は,自然言語で記述された要求から計画を生成し,外部ツールや API を呼び出す能力を備えており,ユーザに形式的な記述を要求することなくサービス合成を行う枠組みとして期待されている。しかし,単一の汎用 LLM エージェントに広範なサービス領域を一括して扱わせる場合,個々のサービス仕様や実行環境に関する専門的な知識の理解が浅くなりやすい。また,多数の API 仕様や利用例を一つのプロンプトに詰め込む必要があるため,プロンプトの膨張とコンテキスト長制限に起因する性能劣化が生じるという問題がある。

本研究では,これらの課題を解決するため,複数の LLM エージェントからなるマルチエージェントシステムに基づくサービス合成手法を提案する。各エージェントを特定の API カテゴリや機能領域に特化させることで,プロンプト内で扱う知識を局所化しつつ,専門性の高いサービス推薦を実現することを目指す。さらに,タスク割当てメカニズムとして契約ネットプロトコルを導入し,マネージャエージェントとコントラクタエージェントがタスク分解と入札・応札を通じて協調的にサービス合成を行う枠組みを設計する。本論文では,マネージャ主導・コントラクタ主導・協調型という三種類のタスク分解プロトコルを比較し,サービス合成精度や単一エージェント方式との性能差,エージェント数や担当 API 数が与える影響を評価することで,LLM ベース・マルチエージェントによるサービス合成の有効性と課題を明らかにする。

そこで本手法の実現にあたり,取り組むべき課題は以下の2点である.
\begin{description}
\item[タスク分解]
自然言語で記述されたユーザ要求から,契約ネットプロトコルに基づいて適切なタスク分解を行う.要求の粒度や曖昧さに応じて,マネージャ主導・コントラクタ主導・協調型といった異なるタスク分解戦略を切り替えつつも,一貫したサービス合成結果を得られる仕組みが必要となる。

\item[コントラクタエージェントの組織化]
エージェントを特定の API カテゴリや機能領域に特化させることでプロンプトの膨張を抑えつつ,サービス仕様や実行環境に関する専門的な知識を十分に反映させる必要がある。また,タスク分解プロトコルの違いがエージェント間の情報共有や推論過程に与える影響を明らかにすることも求められる。
\end{description}

以下,本論文では,第2章で複合サービス合成の基本的なアプローチとして水平型・垂直型サービス合成と,
LLM を用いたサービス合成について説明する.第3章では,契約ネットプロトコルに基づくマルチエージェント方式と
タスク分解プロトコルの設計を示す.第4章では,評価指標と実験設定を述べ,評価結果を示す.第5章で考察を行い,
第6章で結論と今後の展望を述べる.

\newpage
\section{サービス合成}\label{sec-service}
本章では,複合サービスの実現に向けたサービス合成手法について説明する.
複合サービスとは,インターネット上に存在する複数のWebサービスを組み合わせて,新たなサービスのことである.近年では,さまざまなWebサービスがAPIを通じて連携可能となっており,ユーザ要求に応じた柔軟なシステム開発が可能となっている.

複合サービスの代表例としてExpediaを挙げる.Expediaは航空券,宿泊,移動手段など複数のAPIを統合することで,旅行計画・予約を一括提供している.航空券の検索では航空会社情報を提供するAPI,予約確定時には決済API,旅の計画閲覧には地図APIを利用し,単独では実現不可能な利便性を提供している.このような複合サービスを開発するためには,ユーザ要求に応じて必要な機能を提供できるWebサービスを適切に発見する必要がある.しかし膨大な数のWebサービスの中から目的に合致したものを探索することは容易ではない.本章では,サービス合成手法について説明する.

\subsection{サービス合成手法}\label{subsec-service}
従来のサービス合成手法では大きく,水平サービス型と垂直サービス型の2つに分類できる.以下でそれぞれについて説明する.

\subsubsection{水平型サービス合成}\label{subsubsec-service-horizontal}
水平型サービス合成とは,あらかじめ与えられたワークフロー(タスク列)に対し,各タスクを実行可能なWebサービス候補の中から,合成全体として望ましい組合せを選択する枠組みである.Zeng らは,同一の機能を提供するサービスが多数存在する状況を想定し,機能差ではなく QoS(非機能的特性) に基づいてサービスを選別する合成方式を提案し\cite{zeng2004qos}.具体的には,価格・応答時間・可用性などの非機能情報から各サービスのQoSベクトルを定義し,ユーザが与える制約や嗜好(重み付け)を用いて候補を評価する.その上でサービス選択を ローカル最適化 と グローバル最適化 の二通りで扱う.ローカル最適化では,各タスクごとに他タスクとの関係を考慮せず,制約を満たす範囲でスコアが最大となるサービスを選ぶため計算が軽い一方,合成全体のQoSが最適になる保証は弱い.これに対しグローバル最適化では,合成全体のQoSを目的としてタスク間の影響も含めて最適な割当を求めることで,合成結果の品質向上を狙う.

\subsubsection{垂直型サービス合成}\label{subsubsec-service-vertical}
垂直型サービス合成は,人工知能のプランニング技術を用いて,ユーザが指定したゴール状態に到達するための Webサービスの実行系列(合成手順)そのものを生成する手法である.Carman らは,サービス合成を計画問題として捉え,オープンで不確実性を含むWeb環境に適した合成アルゴリズムを提案した\cite{carman2003composition}.提案手法の要点は二つに整理できる.第一に,セマンティック型マッチングである.Webサービス間では,出力と入力のデータ型が厳密に一致しないことが多く,単純なラベル一致に基づく接続判定では合成可能性を過小評価する.そこで,出力型が他サービスの入力型や最終的に達成すべき目標型と意味的に関連するかを,ラベルが指す概念の一般化関係や型構造も含めて判断し,異種スキーマ間の非互換性を緩和する.第二に,探索と実行を統合した逐次的な合成である.サービスの実行結果や入出力値が事前に確定しない状況では,合成計画を最初から完全に構築することが難しいため,候補サービスを選択して実行し,得られた結果を観測して次のステップを更新する手順を繰り返す.このように探索と実行を交互に行うことで,不完全な情報の下でも計画を修正しながら目標達成へ近づく柔軟性を確保する.以上より,本研究は,型の意味的整合性に基づく接続判定と,試行錯誤を許容する逐次実行型の探索戦略を組み合わせ,垂直型サービス合成を実現する先行研究として位置付けられる.

\subsection{LLMを用いたサービス合成}\label{subsec-LLM}
LLMを用いたサービス合成では,自然言語で記述されたユーザ要求から,適切なWebサービスを選択・組み合わせる手法である.従来は機能要件を論理式で記述し,形式的なフロー設計が必要であったが,LLMの自然言語理解能力を活用することで,要求の意図や制約を推定し,それに基づいてタスク分割やAPI選択を行える.また,APIドキュメントの要約や入力パラメータの生成をLLMが補助することで,ユーザは専門知識なしにサービス合成を実現できる.

一方で,LLMを単純にプロンプト利用するだけでは,APIの最新知識不足やハルシネーションにより推薦精度が低下することが指摘されている.さらにBERT等の従来モデルはAPIドメイン固有の知識やマッシュアップの複雑な使用パターンを十分に捉えられず,実運用レベルの推薦に課題が残る.この問題に対し,QinらはLlama-3.2-3Bを基盤とするLLMARを提案し,指示学習と多段階ファインチューニングによってAPIドメイン知識をモデル内部に注入する枠組みを示した.

LLMARの特徴は,API推薦を単一タスクとして扱わず,以下の4つのサブタスクに分解して学習する点にある.
\begin{itemize}
  \item API記述生成(AD): API名から詳細説明を生成し,API機能の理解を深める.
  \item API分類(AC): APIのカテゴリを推定し,属性情報を学習する.
  \item マッシュアップ分類(MC): 要求記述からカテゴリを推定し,構成パターンを獲得する.
  \item API推薦(AR): 要求記述に対してAPI名を直接生成する.
\end{itemize}
これらを同時学習するのではなく,AD/AC/MCで基礎知識を注入した後にARへ段階的に特化させることで,タスク間の表現衝突や負の転移を回避し,汎化性能を高めている.

学習効率の面ではLoRAを採用し,巨大モデル全体を更新せずに低ランク行列のみを学習することで,更新パラメータ数を大幅に削減している.実験ではProgrammableWeb由来のAPI・マッシュアップデータで評価され,ベースモデルや大規模汎用LLMを上回る精度が報告された.これらの結果は,LLMをサービス合成に適用する際,ドメイン知識の構造化と多段階学習が有効であることを示している.

\newpage
\section{契約ネットプロトコルに基づくサービス合成}\label{sec-cnt}
本章では,LLMを用いたマルチエージェントシステムに基づくサービス合成手法について説明する.マルチエージェントを用いる中で本研究では契約ネットプロトコル(Contract Net Protocol; CNP)を採用し,エージェント間の協調メカニズムとして利用する.

\subsection{契約ネットプロトコル}\label{subsec-cnt}

契約ネットプロトコル(Contract Net Protocol; CNP)は,
Reid G. Smith によって提案された分散問題解決のための高水準通信プロトコルであり,
分散環境におけるタスク割当てと協調計算を,交渉に基づくメッセージ交換によって実現する枠組みである.
CNP では,タスクを外部に委託したい側のエージェントをマネージャ(manager),タスクの実行を引き受ける側のエージェントをコントラクタ(contractor)と呼び,両者の間でタスクの告知,入札,契約締結,結果報告といった
一連のやり取りを明示的なプロトコルとして規定している.

典型的な CNP の相互作用は,次のようなステップで構成される.
\begin{enumerate}
  \item {タスク告知(task announcement / call for proposals)}\\
        マネージャは,自身が保持するタスクの内容(目的,前提条件,必要資源,締切,評価基準など)を記述した
        告知メッセージをブロードキャストし,コントラクタ候補に対して提案を呼びかける.
  \item {入札(bidding / proposal submission)}\\
        告知を受け取った各コントラクタは,タスクを遂行可能かどうか,現在の負荷や利用可能資源を考慮して評価する.
        実行可能であれば,予想コスト,所要時間,達成可能な品質などを含む入札(プロポーザル)をマネージャに送信し,
        実行困難と判断した場合は辞退メッセージを返す.
  \item {契約締結・割当て(award / contract)}\\
        マネージャは,収集した複数の入札を比較し,評価基準に基づいて最も望ましいコントラクタを選定する.
        選定されたコントラクタには契約成立を示すアワードメッセージを送り,
        その他のコントラクタには不採択を通知する.これにより,当該タスクに関する
        一時的なマネージャ–コントラクタ関係が形成される.
  \item {実行と結果報告(task execution / status reporting)}\\
        コントラクタは割り当てられたタスクを実行し,中間状態や完了結果をステータス報告メッセージとしてマネージャに送信する.
        マネージャはこれらの報告に基づいてタスクの進捗を把握し,必要であれば再割当てなどの制御を行う.
\end{enumerate}

このように,CNP は単なるタスク割当てアルゴリズムというよりも,
タスク告知から結果報告に至るまでのエージェント間通信の流れとメッセージ種別を明確に定義したコミュニケーション・プロトコルとして位置づけられる.
各エージェントは,あるタスクに関してはマネージャとして振る舞い,別のタスクに関してはコントラクタとして振る舞うことができるため,
システム全体としては固定的な中央管理者を持たない柔軟な階層構造を動的に形成できるという特徴を持つ.
また,タスク告知の範囲や入札の条件を調整することで,探索空間の絞り込みや通信量の制御が行える点も利点とされている.
本研究では,この CNP をエージェント間の協調メカニズムとして採用し,
サービス合成におけるタスク(機能分割や API 選択など)の割当てに応用する.

\begin{figure}[!b]
  \centering
  \includegraphics[width=0.6\linewidth]{image/cnp.png}
  \caption{CNPのシーケンス図}
  \label{fig:cnp_image}
\end{figure}

\subsection{システム概要}\label{subsec-system}
本研究で提案するサービス合成システムは,LLM を用いたマルチエージェントシステムに基づき,
自然言語で与えられた複合サービスの要件から,利用すべき Web サービス API の組み合わせを推薦することを目的としている.

従来の LLM による API 推薦では,要件理解から API 選択までを単一の推論として一括に処理する設計が多い.
しかしこの設計では,多数のAPI説明文や仕様を単一プロンプトへ集約する必要があり入力が肥大化しやすい,そして要件や API 群の変動により推論の再現性が低下しやすい.さらに推論過程がブラックボックス化し,誤りが生じた際にどの段階で破綻したかを切り分けにくいといった課題が生じる.
各ステップに目的・参照情報・出力形式を与えて推論を誘導する Guided Reasoning Chains の考え方に基づき,
サービス合成推論を LLM の内部処理に閉じず,外部化可能な推論タスクの系列として捉える.


本システムは,ユーザ要件を受理しタスクを編成するマネージャエージェント,
各 API カテゴリや個別 API に専門化した複数のコントラクタエージェント,
および各エージェントが参照する API 説明文から構成される.
エージェント間の協調枠組みとして Contract Net Protocol(CNP)を採用し,
タスクの公告(Call for Proposals),入札(Proposals),選択(Award)という流れのもとで推論タスクを分担し,
最終的な API 推薦結果を得る.
CNP を採用する理由は以下の通りである.
\begin{enumerate}
  \item {専門性に基づく自律的な可否判断を反映するためである.}\\
  本研究では,コントラクタエージェントを API カテゴリまたは個別 API に専門化させ,
  各エージェントが保持する局所的知識に基づいて「担当可能性」と「候補 API」を判断する設計をとる.
  CNP は,公告に対して各コントラクタが自律的に入札・辞退を返す手順を持つため,
  専門性に依存した可否判断を自然に取り込める.
  \item {複数候補を比較可能な形で収集し,最終決定を体系化するためである.}\\
  サービス合成では,要求に適合する API 候補が複数存在し得るため,
  候補の列挙に留まらず,根拠や評価観点を揃えて比較し,最終的な採択を行う必要がある.
  CNP は複数の入札を収集し,マネージャが採択する流れを規定するため,
  候補の収集から統合・決定までを一貫した枠組みとして扱える.
  \item {責任分担の差分をプロトコルとして明示し,方式間の比較を可能にするためである.}\\
  本研究の焦点は,推論タスクの責任分担(どの推論ステップを誰が担当するか)が,
  推薦精度に与える影響を明らかにする点にある.
  CNP は公告・入札・選択という役割とメッセージの流れを明確に定義しているため,
  責任分担の差分をプロトコルとして記述でき,方式間の比較・検証を体系的に行える.
\end{enumerate}

その上で,本システムにおけるサービス合成の推論過程を,
以下の 4 ステップからなる共通フローとして定義する.
\begin{enumerate}
  \item コア機能分解:要件文からユーザが最終的に達成したい目的を推定し,必要となる機能単位を抽出する.
  \item カテゴリマッピング:抽出した機能を実現し得る API カテゴリ(例:決済,地図,予約管理など)を推定する.
  \item API マッチング:推定カテゴリに属する API 説明文と要件に基づいて,適合する候補 API を抽出する.
  \item API 選択:抽出した候補 API を比較し,要件適合性の観点から採用 API を確定する.
\end{enumerate}

ここで,(3) は候補 API 集合の抽出(候補提示),(4) は候補集合からの最終採択として区別する.

この 4 ステップは単一のエージェントが一括で実行するのではなく,
CNP による公告・入札・選択の枠組みにおいて,マネージャとコントラクタの間で分担される推論タスクとして扱われる.
マネージャは要件文の受理,推論タスクの定義と公告,入札結果の集約と整合性確認,およびAPI群の確定を担う.
コントラクタは担当の API 知識に基づき,公告されたタスクに対する判断と候補 API,および根拠を入札として提示する.

本研究では,推論タスクの責任分担(どのステップを誰が担当するか)を CNP 上の設計変数とみなし,
マネージャ主導型,コントラクタ主導型,協調型の 3 方式を設計・比較する.
この比較により,推論の分割粒度と専門化の度合いが,API 推薦精度および推論コストに与える影響を明らかにする.


\subsection{LLMに基づくエージェント}\label{subsec-llm-agent}
本研究では,自然言語で記述されたユーザ要求および API 説明文を入力として,
サービス合成のためのAPI群を生成する推論主体として,LLMに基づくエージェントを用いる.

本システムにおける LLM エージェントは,
マネージャエージェントと複数のコントラクタエージェントから構成される.
各エージェントは,CNP に基づく公告・入札・選択の枠組みの中で,
自身に割り当てられた推論タスクを実行し,
その結果を明示的なメッセージとして外部に出力する.
これにより,サービス合成推論は LLM の内部推論として閉じるのではなく,
外部から観測・比較可能な推論タスクの系列として実現される.


\subsubsection{コントラクタエージェント}\label{subsubsec-bidder-agent}
コントラクタエージェントは,個別 API に専門化された LLM エージェントであり,
マネージャから公告されたタスク仕様を入力として受け取る.
各コントラクタは,自身が参照可能な API 説明文のみを用いて推論を行い,
当該タスクを担当可能か否かを判断する.

担当可能と判断した場合,
コントラクタエージェントは,適合すると考えられる API 候補と,
その判断根拠を入札としてマネージャに返す.
ここで提示される API 候補は,
システム概要で定義した推論ステップ (3) API マッチングに相当し,
候補 API 集合の抽出結果として位置付けられる.
一方,担当不可能と判断した場合には,
辞退メッセージを返すことで入札を行わない.


\subsubsection{マネージャエージェント}\label{subsubsec-manager-agent}
マネージャエージェントは,
ユーザ要求を受理し,サービス合成全体を統括する役割を担う LLM エージェントである.
マネージャは,システム概要で定義した推論過程に基づいて,
推論タスクを構成し,CNP に従ってコントラクタエージェントへ公告する.

入札を受信した後,マネージャエージェントは,
複数のコントラクタから提示された API 候補と根拠を比較・統合し,
要求全体との整合性を確認した上で,
最終的に採用する API 集合を確定させる.
この処理は,推論ステップ (4) API 選択に対応し,
候補集合からの最終採択を意味する.

このように,マネージャエージェントは,
分散した推論結果を集約する集中意思決定の役割を担い,
コントラクタエージェントは専門性に基づく局所的推論を担うことで,
分散推論と統合的意思決定を両立させている.


\subsection{プロトコル}\label{subsec-Protocol}
本研究では,CNP に基づくサービス合成において,
推論タスクの責任分担(どの推論ステップをどのエージェントが担当するか)を
設計変数として扱う.
具体的には,システム概要で定義した 4 ステップの推論過程
(コア機能分解,カテゴリマッピング,API マッチング,API 選択)を,
マネージャエージェントとコントラクタエージェントの間で
どのように分担させるかによって,
異なるサービス合成プロトコルを設計する.

本節では,推論タスクの分担方針が異なる
マネージャ主導型,コントラクタ主導型,協調型の 3 方式を定義し,
各方式における公告内容,入札内容,および最終的なAPI群の確定方法を明確にする.

\subsubsection{マネージャ主導型タスク分解}\label{subsubsec-manager-main}
マネージャ主導型では,サービス合成に必要な推論の大部分を
マネージャエージェントが担当する.
4 ステップの推論過程における役割分担は以下の通りである.

\begin{itemize}
  \item {マネージャの担当:}
  (1) コア機能分解,(2) カテゴリマッピング,(4) API 選択
  \item {コントラクタの担当:}
  (3) API マッチング
\end{itemize}

本方式では,マネージャエージェントがユーザ要求を解析し,
必要なコア機能および対応する API カテゴリを事前に確定した上で,
それらをタスク仕様としてコントラクタに公告する.
公告には,対象となる機能,想定カテゴリが含まれる.

各コントラクタエージェントは,
公告された機能及びカテゴリに基づき,
自身が参照可能な API 説明文を用いて API マッチングを行い,
適合すると判断した API 候補(自身)とその根拠を入札として返す.
マネージャは,収集した入札を比較・統合し,
要求全体との整合性を確認した上で,
最終的な API 集合を確定させる.

\begin{figure}[!b]
  \centering
  \includegraphics[width=\linewidth]{image/manager.png}
  \caption{マネージャ主導プロトコルのシーケンス図}
  \label{fig:manager}
\end{figure}


\subsubsection{コントラクタ主導型タスク分解}\label{subsubsec-bidder-main}
コントラクタ主導型では,
推論の大部分をコントラクタエージェントに委ねる.
4 ステップの推論過程における役割分担は以下の通りである.

\begin{itemize}
  \item {マネージャの担当:}
  (4) API 選択
  \item {コントラクタの担当:}
  (1) コア機能分解,(2) カテゴリマッピング,(3) API マッチング
\end{itemize}

本方式では,マネージャエージェントは,
ユーザ要求文そのものをタスクとして公告し,
詳細な分解やカテゴリ推定は行わない.
各コントラクタエージェントは,
自身の担当 API 知識に基づいて要求文を解釈し,
コア機能分解およびカテゴリマッピングを行った上で,
適合する API 候補と根拠を入札として提示する.

マネージャエージェントは,
複数のコントラクタから提出された入札を集約し,
提案内容の整合性を確認した上で,
最終的に採用する API 集合を確定させる.

\begin{figure}[!b]
  \centering
  \includegraphics[width=\linewidth]{image/contractor.png}
  \caption{コントラクタ主導プロトコルのシーケンス図}
  \label{fig:contractor}
\end{figure}

\subsubsection{協調型タスク分解}\label{subsubsec-cooperative}
協調型では,
マネージャとコントラクタが推論タスクを分担し,
両者の役割を組み合わせる.
4 ステップの推論過程における役割分担は以下の通りである.

\begin{itemize}
  \item {マネージャの担当:}
  (1) コア機能分解
  \item {コントラクタの担当:}
  (2) カテゴリマッピング,(3) API マッチング,(4) API 選択
\end{itemize}

本方式では,マネージャエージェントが
ユーザ要求からコア機能分解を行い,
抽出した機能単位をタスクとして公告する.
各コントラクタエージェントは,
公告された機能に対してカテゴリ候補を推定し,
担当 API 知識に基づいて API マッチングおよび選択を行い,
採用すべき API 候補と根拠を入札として返す.

マネージャは,
各入札を統合し,
要求全体との整合性を確認した上で,
最終的な API 集合をAPI群として確定させる.

\begin{figure}[!b]
  \centering
  \includegraphics[width=\linewidth]{image/cooperate.png}
  \caption{協調型プロトコルのシーケンス図}
  \label{fig:cooperate}
\end{figure}

\newpage
\section{評価}\label{sec-evaluation}

\subsection{実験設定}\label{subsec-format}
本研究では,提案する CNP ベースのマルチエージェント方式について,
推論タスク割り当てプロトコル(マネージャ主導・コントラクタ主導・協調型)と
コントラクタエージェントの組織化(単一 API 単位/カテゴリ単位)を比較する.
いずれの設定においても,ユーザ要求(マッシュアップ要件)を入力としてAPI群を生成し,
正解計画との照合によって計画生成精度を評価する.
LLM には gpt-oss20B を用いた.

\subsubsection{実験データ}\label{subsubsec-data}
本研究では,サービス推薦モデルを構築するために,ProgrammableWeb に掲載されている
複合サービス(Mashup)データおよび原子サービス(Web API)データを使用した.
複合サービスデータは,複合サービス名をキーとし,複合サービスカテゴリ,複合サービス説明文,
構成サービス(使用 API の一覧),および ID を値として保持する.複合サービスカテゴリおよび構成サービスは
複数存在する場合があるため,いずれもリスト型として扱う.
原子サービスデータは,原子サービス名をキーとし,サービスカテゴリ,原子サービス説明文,
サービス提供者,サービス利用者,および ID を値として保持する.カテゴリ情報は複数付与されており,
そのうち先頭要素をメインカテゴリ,それ以外をサブカテゴリとして扱う.
これらの属性はいずれも ProgrammableWeb 上でユーザにより付与されたものである.
複合サービスデータには欠損が含まれるため,前処理としてフィルタリングを行った.
フィルタリング条件は,複合サービスが参照する構成サービス(使用 API)が,原子サービスデータに存在しないレコードを除外することである(データ構造は図2・図3に示す).この処理の結果,複合サービスデータは 4284 件となった.そして原始サービス数は

なお,本節では実データ例ではなく,データ構造の構成概念(スキーマ)を図として示す.
原子サービス(Web API)データの構成概念を図\ref{fig:api},
複合サービス(Mashup)データの構成概念を図\ref{fig:mashup} に示す.

また,複合サービスデータには欠損が含まれるため,
複合サービスの構成サービス(使用 API)が原子サービスデータに含まれないレコードを除外するフィルタリングを行った.

その中で複合サービス100件を抽出し,評価データセットとして用いた.

\begin{figure}[htbp]
  \centering
  \includegraphics[width=0.5\linewidth]{image/api_info.png}
  \caption{原子サービスデータの構成概念}
  \label{fig:api}
\end{figure}

\begin{figure}[htbp]
  \centering
  \includegraphics[width=0.5\linewidth]{image/mushup_info.png}
  \caption{複合サービスデータの構成概念}
  \label{fig:mashup}
\end{figure}



\subsubsection{評価指標}\label{subsubsec-metrics}
提案手法(CNP に基づくマルチエージェント方式)で生成される API 推薦結果の品質を評価するため,
適合率(Precision),再現率(Recall),および F1 スコアを用いる.
照合は API 名一致に基づく集合評価として行い,正解集合を $G$,生成された推薦 API 集合を $P$ とすると,
$TP, FP, FN$ はそれぞれ以下で定義される.
\begin{equation}
TP = |P \cap G|,\quad FP = |P \setminus G|,\quad FN = |G \setminus P|
\end{equation}
ここで,$TP$(True Positive)は推薦した API のうち正解に含まれる API 数,
$FP$(False Positive)は推薦したが正解に含まれない API 数,
$FN$(False Negative)は正解に含まれるが推薦できなかった API 数を表す.
このとき,適合率,再現率,F1 スコアは以下の式から算出される.
\begin{equation}
\text{適合率} = \frac{TP}{TP + FP}
\end{equation}
\begin{equation}
\text{再現率} = \frac{TP}{TP + FN}
\end{equation}
\begin{equation}
\text{F1 スコア} = 2 \cdot \frac{\text{適合率} \cdot \text{再現率}}{\text{適合率} + \text{再現率}}
\end{equation}

本研究では,最終的な推薦結果だけでなく,プロトコルの各段階における中間結果についても同様の指標を算出し,
推論過程のどの段階で精度が変化するかを分析する.
具体的には,各プロトコルにおいて以下の段階で評価を行う.

\begin{description}
\item[カテゴリ割当て時点]
マネージャまたはコントラクタがユーザ要求に対して API カテゴリを推定した段階である.
推定されたカテゴリに属する全 API を候補集合 $P$ として,正解集合 $G$ との照合を行う.
この段階の評価により,カテゴリレベルでの要求理解の精度を把握できる.

\item[入札時点]
コントラクタエージェントが入札として提示した API 候補の集合を $P$ として評価する.
この段階では,各コントラクタが自身の担当 API 知識に基づいて選定した候補が含まれるため,
カテゴリ割当て時点よりも候補が絞り込まれる.
入札時点の評価により,コントラクタの専門的知識がどの程度活用されているかを確認できる.

\item[最終統合後]
マネージャエージェントが全コントラクタからの入札を集約・統合し,
最終的に採用する API 集合を確定した段階である.
この段階の評価が,提案手法全体としての推薦精度を表す.
\end{description}

これら 3 段階の評価を比較することで,
各プロトコルにおいて推論のどの段階で精度が向上または低下するかを明らかにし,
タスク分解方式の違いが推薦精度に与える影響を詳細に分析する.

\subsubsection{マネージャ主導タスク分解}\label{subsubsec-manager-conclusion}
マネージャ主導方式の結果を示す.
本方式では,まずマネージャが要件文をタスク分解し,各タスクに対してカテゴリを割り当てる.その後,要件文とカテゴリを対象カテゴリのコントラクタエージェントに提示して入札を受け,最後に入札結果から適切な API を統合するという流れである.
表~\ref{tab:manager-results}に示す通り,カテゴリ割当て時点では適合率は 0.111,再現率は 0.607,F1 値は 0.177 となった.
入札時点では適合率は 0.047,再現率は 0.485,F1 値は 0.079 となった.適合率は前の時点に比べて低下した.これは,入札で候補APIが拡張され,正解に含まれない候補も多く含まれるためである.
最終統合後では適合率は 0.159,再現率は 0.298,F1 値は 0.193 となった.適合率は前の時点に比べて改善した.

\begin{table}[tb]
  \centering
  \caption{マネージャ主導方式の推薦結果}
  \label{tab:manager-results}
  \begin{tabular}{l c c c}
    \hline
    段階 & 適合率 & 再現率 & F1 値 \\
    \hline
    カテゴリ割当て時点 & 0.111 & 0.607 & 0.177 \\
    入札時点 & 0.047 & 0.485 & 0.079 \\
    最終統合後 & 0.159 & 0.298 & 0.193 \\
    \hline
  \end{tabular}
\end{table}

\subsubsection{コントラクタ主導タスク分解}\label{subsubsec-bidder-conclusion}
コントラクタ主導方式の結果を示す.コントラクタが担当領域の API 仕様に基づいて分解を提案するため,
局所的な知識を反映した提案が得られる一方で,提案間の統合にはマネージャ側での整合性確認が重要となる.
\subsubsection{協調型タスク分解}\label{subsubsec-cooperative-conclusion}
協調型方式の結果を示す.マネージャによるタスク分解とコントラクタによるカテゴリ分解を組み合わせることで,
整合性と柔軟性の両立を図る.
表~\ref{tab:cooperation-results}に示す通り,カテゴリ割当て時点では適合率は 0.047,再現率は 0.347,F1 値は 0.078 となった.
入札時点では適合率は 0.034,再現率は 0.260,F1 値は 0.058 となった.適合率は前の時点に比べて低下した.これは,入札で候補APIが拡張され,正解に含まれない候補も多く含まれるためである.
最終統合後では適合率は 0.083,再現率は 0.223,F1 値は 0.114 となった.適合率は前の時点に比べて改善した.
また,最終統合後の精度はマネージャ主導方式よりも低く,分担により柔軟性は高まる一方で,
提案の統合段階における情報損失が影響していると考えられる.

\begin{table}[tb]
  \centering
  \caption{協調型方式の推薦結果}
  \label{tab:cooperation-results}
  \begin{tabular}{l c c c}
    \hline
    段階 & 適合率 & 再現率 & F1 値 \\
    \hline
    カテゴリ割当て時点 & 0.047 & 0.347 & 0.078 \\
    入札時点 & 0.034 & 0.260 & 0.058 \\
    最終統合後 & 0.083 & 0.223 & 0.114 \\
    \hline
  \end{tabular}
\end{table}

\newpage
\section{考察}
\label{sec-discussion}

\subsection{単一エージェントとの比較}\label{subsec-comparison}
単一の汎用 LLM エージェントに多数の API 仕様を集約する方式では,プロンプト肥大化と文脈長制約により
推論精度が低下しやすい.一方で,マネージャ主導方式は,マネージャがタスク分解とカテゴリ割当てを行い,
コントラクタが担当領域の API 仕様に基づいて入札し,マネージャが整合性を確認して統合するという
段階的なプロトコルを持つ.この分担により,単一エージェント方式で生じやすい
「過剰に広い候補列挙」や「曖昧要求に対する一括推論」の影響が抑えられ,
担当領域の文脈を局所化した提案が集約されるため,再現率と適合率の両面で改善が見られた.
実際に,単一エージェント方式は適合率 0.137,再現率 0.141 であるのに対し,
マネージャ主導方式は適合率 0.159,再現率 0.298 を示しており,
タスク分解と入札・統合のプロトコル差が結果の差に寄与していると考えられる.

\begin{table}[tb]
  \centering
  \caption{単一エージェント方式の推薦結果}
  \label{tab:single-results}
  \begin{tabular}{l c}
    \hline
    指標 & 値 \\
    \hline
    適合率 & 0.137 \\
    再現率 & 0.141 \\
    F1 値 & 0.130 \\
    \hline
  \end{tabular}
\end{table}

\subsection{今後の課題}\label{subsec-future-work}
今後の課題として,(1) 公告範囲の動的制御や入札の事前フィルタリング等による通信効率化,
(2) 要求タイプ推定に基づくプロトコル自動選択,(3) Plan の入出力整合性検査の強化(型・スキーマの導入)
などが挙げられる.

\newpage
\section{おわりに}
\label{sec-conclusion}
本論文では,LLM を用いたマルチエージェントシステムに基づく Web サービス合成手法を提案した.
契約ネットプロトコルに基づくタスク分解メカニズムを導入し,マネージャ主導・コントラクタ主導・協調型という
三種類のタスク分解プロトコルを設計・比較した.また,入札エージェントの組織化として,単一 API 単位の割当てに加え,
カテゴリ単位で複数 API をまとめて担当させる設計を比較した.

本研究の貢献は以下である.本研究で設定した課題に対して,次のように具体的な検証と知見を示した.
\begin{description}
\item[推論タスク割り当て]
マネージャ主導・コントラクタ主導・協調型の 3 方式を比較し,要求の粒度・曖昧さに応じて有利な方式が異なることを示した.
要求タイプに応じて方式を切り替える運用が,単一方式に固定するよりも有効であることを確認した.
\item[コントラクタエージェントの組織化]
API 仕様を分散保持する専門エージェントの設計を評価し,単一エージェント方式と比較して
推薦の正確性を高めつつ,大規模サービス集合に対してもスケーラブルに動作することを示した.
\end{description}

今後の展望として,通信効率化,プロトコル自動選択,Plan の入出力整合性検査の強化などに取り組むことで,
より実用的で柔軟なサービス合成システムの実現を目指す.

\newpage
\bibliographystyle{kuisunsrt}
\nocite{*}
\bibliography{references}
\end{document}
